\newpage
\section{电荷}

\vspace{10pt}
\begin{itemize}
\item 物体能够吸引轻小物体,就说物体带了电,即物体带了\blue{电荷}(electric charge). 带了电荷的物体叫做\blue{带电体}.
\item 使物体带电叫做\blue{起电}. 用摩擦的方式使物体带点叫做\blue{摩擦起电}(electrification by friction).
\item 自然界\blue{只有}两种电荷.
\item 用丝绸摩擦过的玻璃棒带的电荷叫做\blue{正电荷}(positive charge). 用毛皮摩擦过的橡胶棒带的电荷叫做\blue{负电荷}(negative charge).
\item \blue{同种}电荷相互\blue{排斥},\blue{异种}电荷相互\blue{吸引}.
\item 电荷的多少叫做\blue{电荷量}(electric quantity),简称\blue{电量}. 用 \blue{$\bm Q$} 或 \blue{$\bm q$} 表示. 在国际单位制重,电荷量的单位是\blue{库仑}(coulomb),简称\blue{库}. 符号是 \blue{$\bm c$}. 正电荷的电荷量为正值,负电荷的电荷量为负值.
\item \blue{验电器}和\blue{静电计}.
\item 两种电荷互相完全抵消叫做\blue{中和}.
\item 物质是由\blue{分子}构成的,分子是由\blue{原子}构成的.
\end{itemize}