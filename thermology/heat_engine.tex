\newpage
\section{热机}

\subsection{热机}
\vspace{10pt}
\begin{itemize}
\item \blue{热机}(heat engine),即利用内能做功(\blue{内能转化为机械能})的机械.
\item \blue{蒸汽机},即利用水蒸气膨胀做功的热机. 蒸汽机属于外燃机.
\item 活塞从气缸的一端运动到另一端的过程叫做一个\blue{冲程}.
\item 四冲程汽油机一般包括\blue{吸气}、\blue{压缩}、\blue{做功}、\blue{排气}四个冲程.
\item \blue{汽油机}和\blue{柴油机}都属于\blue{内燃机}.
\item \blue{汽轮机}和\blue{喷气发动机}.
\end{itemize}

% \newpage
\subsection{热机的效率}
\vspace{10pt}
\begin{itemize}
\item 能够燃烧的物质叫做\blue{燃料}.
\item 在燃烧过程中,燃烧的\blue{化学能}转化为\blue{内能}.
\item 某种燃料\blue{完全燃烧}放出的能量与其质量或体积的比叫做这种燃料的\blue{热值}(combustion value)或燃烧值. 用符号 \blue{$\bm q$} 表示. 单位是\blue{焦耳每千克($\bf J/kg$)}或\blue{焦每立方米($\bf J/m^3$)}. 有:
\mathline{q=\frac{Q_{\tiny\mbox{放}}}m~\mbox{或}~q=\frac{Q_{\tiny\mbox{放}}}V}
\item 热值在数值上等于 \blue{$\bf1kg$} 或 \blue{$\bf1m^3$} 的某种燃料\blue{完全燃烧}放出的热量. 其中 $1\text{m}^3$ 是\blue{标准状态}下气体燃料的体积. 标准状态是指温度为 \blue{$\bf0\textcelsius$}、压强为 \blue{$\bf1\text{atm}$} 的状态.
\item 热量的计算有 \blue{$\bm{Q_{\tiny\mbox{放}}=qm}$} 或 \blue{$\bm{Q_{\tiny\mbox{放}}=qV}$}.
\item 做有用功的能量与燃料完全燃烧放出的能量之比叫做\blue{热机的效率},有:
\mathline{\eta=\frac{Q_{\tiny\mbox{有用}}}{Q_{\tiny\mbox{燃料}}}\cdot100\%}
\item 设燃料放出的热量为 $Q_1$,热机吸收的热量为 $Q_2$,废气带走的热量为 $Q_3$,则:
$$
\eta=\frac{Q_1-Q_2-Q_3}{Q_1}\cdot100\%
$$
\end{itemize}