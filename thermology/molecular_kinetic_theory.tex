\section{分子动理论}

\subsection{分子动理论(molecular kinetic theory)}
\vspace{10pt}
\begin{itemize}
\item 物质是由大量分子组成的.
\item 分子在\blue{永不停息}地做\blue{无规则}运动.
\item 分子之间存在相互作用力.
\item 分子直径约为 \blue{$\bf10^{-10}m$}.
\item 18g 水中含有水分子的个数约为 \blue{$\bf6.02\times10^{23}$},即为阿伏伽德罗常数 $N_{\text{A}}$.
\item 在研究物体的热运动性质和规律时,不必区分它们在化学变化中所起的不同作用,而把组成物体的微粒统称为\blue{分子}(molecule).
\item 不同的物质在相互接触时\blue{自发地}彼此进入对方的现象叫做\blue{扩散}(diffusion).
\item 扩散现象可以发生在气体、液体和固体之间.
\item 扩散现象是物质分子永不停息地做无规则运动的证据之一.
\item 悬浮微粒的无规则运动叫做\blue{布朗运动}(Brownian motion).
\item \blue{悬浮微粒的无规则运动并不是分子的运动},但可以间接地反应液体分子运动的无规则性.
\item 分子的无规则运动叫做\blue{热运动}(thermal motion). 
\item \blue{温度}是\blue{分子热运动剧烈程度}的标志.
\item \blue{分子之间存在引力},\blue{分子之间存在斥力}.
\item 分子之间,引力和斥力\blue{同时存在}.
\item 分子间的作用力 $F$ 与分子间距离 $r$ 有关. 即:
\newline 当 $r=r_0$ 时,分子间的作用力 $F$ 为 0,这个位置被称为\blue{平衡位置}.
\newline 当 $r>r_0$ 时,分子间的作用力 $F$ 表现为引力.
\newline 当 $r<r_0$ 时,分子间的作用力 $F$ 表现为斥力.
\begin{figure}[H]
	\centering
	\begin{tikzpicture}
	\begin{axis}[
		axis lines = middle,
		xmin = 0, xmax = 8,
		ymin = -4, ymax = 4,
		smooth, thick,
		xlabel = {$r$}, ylabel = {$F$},
		xlabel style = {anchor = north},
		ylabel style = {anchor = east},
		xtick = {0.83627}, xticklabels = {$r_0$}, 
		ytick = \empty,
		samples = 200,
		legend entries = {\kaishu{引力}, \kaishu{斥力}},
		legend style = {font = \small},
	]
		\addplot+[no marks, domain = 0.5 : 7.5]{6 / (x + 1) ^ 1.3};
		\addplot+[no marks, domain = 0.5 : 7.5]{-5 / (x + 1)};
	\end{axis}
	\node at (0, 2) [below = 5pt, left = -1pt] {$O$};
	\end{tikzpicture}
	~~
	\begin{tikzpicture}
	\begin{axis}[
		axis lines = middle,
		xmin = 0, xmax = 4,
		ymin = -2, ymax = 2,
		smooth, thick,
		xlabel = {$r$}, ylabel = {$F$},
		xlabel style = {anchor = north},
		ylabel style = {anchor = east},
		xtick = {0.46607}, xticklabels = {$r_0$}, xtick style = {anchor = north east},
		ytick = \empty,
		samples = 200,
		legend entries = {\kaishu{合力}},
		legend style = {font = \small},
	]
		\addplot+[no marks, domain = 0.25 : 3.75]{9 / ((x + 0.85) ^ 6) - 3 / ((x + 0.85) ^ 2)};
	\end{axis}
	\node at (0, 2) [below = 5pt, left = -1pt] {$O$};
	\end{tikzpicture}
	\titlename{分子间的作用力与分子间的距离的关系}
\end{figure}
\end{itemize}

\subsection{固体~~液体~~气体}
\vspace{10pt}
\begin{itemize}
\item 固体分子间的距离小,不容易被压缩和拉伸,具有一定的体积和形状.
\item 气体分子间的距离很大,彼此间几乎没有作用力. 具有流动性,容易被压缩.
\item 液体分子间的距离比气体小、比固体大,液体分子间的作用力比固体小、比气体大,分子没有固定的位置,运动较自由. 液体较难被压缩,没有一定的形状,具有流动性.
\begin{table}[H]
	\kaishu
	\centering
	\begin{tabular}{ccccc}
	\toprule
	\multirow{2}*{物态} & \multicolumn{2}{c}{微观特性} & \multicolumn{2}{c}{宏观特性} \\
	& 分子间距离 & 分子间作用力 & 固定形状 & 固定体积 \\ 
	\midrule
	固态 & 很小 & 很大 & 是 & 是 \\ 
	液态 & 较大 & 较大 & 否 & 是 \\ 
	气态 & 很大 & 很小 & 否 & 否 \\ 
	\bottomrule
	\end{tabular}
	\titlename{固体~~液体~~气体}
\end{table}
\end{itemize}